\subsection{Oscillations for cognitive processes}
Within recent years, a number of publications has related local control of cognitive functions in mammalian cortex to the interplay of $\gamma$-range to lower-frequency oscillations \citep{Fries2015,Bastos2015}. 
Experiments suggest that $\gamma$-band activity from primary sources in the supragranular neocortical layers are a neuronal correlate of bottom-up processing of sensory information, whereas top-down control is represented by lower frequency oscillations from primary sources in infragranular layers \citep[Results, 2nd par]{Bollimunta2011a} \citep[p4, upper left]{VanKerkoerle}.
In a series of hierarchically connected areas these differential bottom-up and top-down streams are beneficial, because activation representing an attended stimulus can be selectively prioritised for processing. Given the two distinct frequency bands, a prioritised stimulus can then, for example, be encoded in cross-frequency interactions, such as phase-amplitude coupling \citep[1st only mentions FA-coupling, 2nd describes the processing in streams]{Jensen,Spaak2012,Osipova2008,Jensen2014}.
The aforementioned concept for differential cortical processing streams is now widely accepted and establishes coherence between a number of theories, such as "Communication through coherence" \citep{Fries2005} and "Predictive coding" \citep{Bastos2012}.
The perhaps most prominent example is the hierarchy of mammalian visual cortex \citep{Felleman91_1}, where top-down control is mediated in the $\vartheta$ $(4\mhyphen7\ut{Hz})$ or $\alpha$-band $(8\mhyphen12.5\ut{Hz})$ \citep[focusses on theta, but Bastos 2015 also mentions $\alpha$]{AndreMoraesBastos2015} and coordinated by thalamic nuclei acting as relay stations \citep[last page bottom]{Jensen2014, Saalmann2012}.

\subsection{Mysteries of the lower layers}
While $\gamma$-oscillations in the brain have been investigated for roughly 30 years now \citep{Gray1989}, some mysteries still remain hidden in the lower layers of visual cortex.
Most notably, the frequency bands of oscillations reported for the top-down processing along a given hierarchy vary across recording methods, species and brain regions: 
for example, \cite[abstract]{Sun} report $10\mhyphen15\ut{Hz}$ in rat visual cortex, \cite[p2, bottom]{VanKerkoerle} already yield $5\mhyphen15\ut{Hz}$ in the macaque monkey, \cite[see abstract]{AndreMoraesBastos2015} exclusively talk about the $\beta$ band at $14\mhyphen18\ut{Hz}$, while \cite[abstract]{Buffalo11_11262} and \cite[abstract]{Fries2015} refer to $\vartheta$-rhythms at $6\mhyphen16\ut{Hz}$ in rhesus and $7\mhyphen8\ut{Hz}$ in macaque monkey respectively - all with laminar recordings from visual cortex areas\footnote{For simplicity, we focus on oscillations in the $\vartheta$- $(4\mhyphen7\ut{Hz})$, and $\alpha$-range $(8\mhyphen12.5\ut{Hz})$, which we refer to as \textit{low-frequency oscillations (LFO)} \citep[definition of the frequency bands]{Gerrard2007, VanAlbada2009}.}.
The variability of oscillation frequencies across brain regions (e.g. exclusively $\beta$ in motor cortex \citep{Schoeffelen2005}) suggests that local, dynamic properties of cortex govern the expression of LFOs. 
In consequence we first investigated LFOs in models, such as the local cortical microcircuit \citep{Potjans}\red{maybe first cite some review about microcircuit here?}.
We know from previous simulation studies \citep{Potjans} that, other than $\gamma$-range oscillations, LFOs do not emerge in biologically informed networks of leaky integrate-and-fire neurons - 'informed' pertaining to external input, number of neurons, synapses, scaling and connectivity in a cortical microcircuit. 
This holds true, when extending the simulation to the full visual hierarchy with all areas represented by appropriately scaled and interconnected microcircuits \citep{Schmidt2015a}.

\subsection{A choice of models}
More recently differential high- and low-frequency oscillations in visual cortex were successfully simulated \citep{Helmer2015, Mejias} in similarly constructed biologically informed multi-scale models of visual cortex. 
Mejias and colleagues recreated the differential oscillatory activation that produce the Granger-causal influence along the visual hierarchy first experimentally shown by \citet{VanKerkoerle} and \citet{AndreMoraesBastos2015}. A central prediction from the study is that inhibitory chandelier cells and irregular spiking interneurons that are targeted by pyramidal neuron projections from layer 5 control the phase-amplitude coupling of $\alpha$- and $\gamma$-oscillations within one local microcircuit.
Helmer and colleagues were able to also reproduce other cross-frequency interactions that can be observed in biological brain networks and exhibit control over a dynamic hierarchical placement of areas in terms of phase delays. They propose that layer-specific expression of oscillatory frequencies arises from inter-layer interactions. \red{more details here?}
However, neither study explains the how a specific intrinsic firing rate resonance is achieved by a network layer, nor the way it is dynamically configured to selectively gate processing by changing the phase delay relative to other areas. 
Indeed, we were primarily interested in the expression of such intrinsic lower frequency oscillations and their phase relation across areas.
To this end, the resonance arising in a smaller version of the network by \citet{Schmidt2015a}, modified at the level of the single-neuron model, was investigated. 
Aside from Mejias' and colleagues proposal it was, however, unclear how the differential frequency bands were realised in the local microcircuit.

\subsection{Subthreshold resonance and resulting network responses}
Although population rate models may capture realistic differential oscillations in frequency bands that are associated with feedforward and feedback processing, a smaller spatial scale is required to explain the frequency preferences of involved oscillating populations. 
In particular, a number of experimental studies has associated resonance at network level to subthreshold resonance of the membrane voltage of single neurons \citep{Hutcheon1996a}, which also shows preferred frequencies \citep[for a review see]{Hutcheon2000}. 
Biological subthreshold membrane potential resonance arises from voltage-gated ionic currents with slow time scales \citep{Ulrich2002a,White1995}. 
By manipulation of some variables governing the ion channels responsible for these slow time scales, a combined experimental and modelling study recently modified frequencies of network level resonance: shifting the peak frequency of subthreshold resonance implicitly also shifted the respective peak in network power spectrum\citep[methods]{Chen2016a}.
The ion channels reconfigured to achieve this shift are located on apical dendrites of pyramidal neurons in the cortical layer 5, which, in turn, are commonly associated with LFOs in the $\vartheta$- and lower $\alpha$-range \citep{Schmidt2016,Hutcheon1996}.
Thus we tested whether $\alpha$-oscillations, that did not occur the in the microcircuit \citep{Potjans} and multi-area model \citep{Schmidt2015a} naturally, would arise when neurons with appropriately configured subthreshold resonance preference were introduced to the model.
As animal studies indicated that up to 50\% of the excitatory pyramidal population in layer 5 expresses subthreshold resonance at LFOs \citep{Silva1991}, we used this as a ballpark number for the size of the excitatory subpopulation in layer 5 of the Potjans- and Schmidt-models.
The implementation of subthreshold resonance itself was achieved by replacing the respective LIF-neurons in the models by \textit{generalised integrate-and-fire neurons with two membrane variables (GIF2)}, as proposed by \cite{Richardson2003a}. 
Other than in the study by \citet{Mejias} et al., our model further included a thalamic population to investigate the relative amplitude in $\alpha$-drive to a visual area required to elicit propagation of the oscillation to neighbouring areas.